\frenchspacing
\documentclass{amsart}
\usepackage{amssymb}
\usepackage{fancyhdr}
\usepackage{bbm}
\usepackage{enumerate}
\usepackage{graphicx}
\usepackage{tikz}
\pagestyle{fancy}
\usepackage[margin=1in]{geometry}
\newgeometry{left=1.5cm, right=1.5cm, top = 1.5cm}
\fancyhfoffset[E,O]{0pt}
\allowdisplaybreaks


\rhead{Andrew Lys}   %% <-- your name here
\chead{Problem Set 8}
\cfoot{\thepage}
\lhead{\today}



%% your macros -->
\newcommand{\nn}{\mathbb N}    %% naturals
\newcommand{\zz}{\mathbb Z}    %%integers
\newcommand{\rr}{\mathbb R}    %% real numbers
\newcommand{\cc}{\mathbb C}    %% complex numbers
\newcommand{\ff}{\mathbb F}
\newcommand{\qq}{\mathbb Q}
\newcommand{\cA}{\mathcal{A}}
\newcommand{\cP}{\mathcal{P}}
\newcommand{\cN}{\mathcal{N}}
\newcommand{\cM}{\mathcal{M}}
\newcommand{\cB}{\mathcal{B}}
\newcommand{\limn}{\lim_{n \to \infty}} %%lim n to infty shorthand
\newcommand{\va}{\mathbf{a}}
\newcommand{\vb}{\mathbf{b}}
\newcommand{\vc}{\mathbf{c}}
\newcommand{\vx}{\mathbf{x}}
\newcommand{\vy}{\mathbf{y}}
\newcommand{\vv}{\mathbf{v}}
\newcommand{\vw}{\mathbf{w}}
\newcommand{\vu}{\mathbf{u}}
\DeclareMathOperator{\Var}{Var}  %% variance
\DeclareMathOperator{\Aut}{Aut}
\DeclareMathOperator{\Sym}{Sym}
\DeclareMathOperator{\Cov}{Cov}
\newtheorem{theorem}{Theorem}
\theoremstyle{definition}
\newtheorem{definition}{Definition}
\newtheorem{exercise}{Exercise}[section]
\renewcommand{\thesubsection}{\arabic{subsection}}

\begin{document}
\noindent
Problem Set 8\hfill \today  %% <-- Update Notes here ***
\smallskip
\hrule
\smallskip
\noindent
Solutions by {\bf Andrew Lys} \qquad   %% <-- your name here ***
  {\tt andrewlys(at)u.e.}      %% <-- your uchicago email address here ***

\vspace{0.5cm}
\subsection{Feature Selection}
\begin{enumerate}[(a)]
  \item 
    \begin{enumerate}[i.]
      \item 
        Let $k = 1$. Then since $x_1$ and $x_{100}$ are uncorrelated, we simply pick our feature as the feature which contributes more to the signal, namely $x_100$. 
        Our predictor is then 
        \[h_w(x) = a x_{100}\]
        And we pick $a = \frac{3}{\sqrt{10}}$.
        Our error is then:
        \[L_D(h_w) = E[x_1^2/10] = \frac{1}{10}\Var(x_1) = \frac{1}{10}\]

        For $k = 2$ and above, we simply pick $x_1$ and $x_{100}$ as our features, $w = \frac{3}{\sqrt{10}} e_{100} + \frac{1}{\sqrt{10}} e_1$. 
        We get zero loss. 
    \end{enumerate}
\end{enumerate}
\subsection{Boosting as Coordinate Descent}
\end{document}
